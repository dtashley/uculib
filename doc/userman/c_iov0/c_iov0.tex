\chapter{Introduction and Overview}
\label{ciov0}

%%%%%%%%%%%%%%%%%%%%%%%%%%%%%%%%%%%%%%%%%%%%%%%%%%%%%%%%%%%%%%%%%%%%%%%%%%%%%%%
%%%%%%%%%%%%%%%%%%%%%%%%%%%%%%%%%%%%%%%%%%%%%%%%%%%%%%%%%%%%%%%%%%%%%%%%%%%%%%%
%%%%%%%%%%%%%%%%%%%%%%%%%%%%%%%%%%%%%%%%%%%%%%%%%%%%%%%%%%%%%%%%%%%%%%%%%%%%%%%
\section{Overview of the Library}
\label{ciov0:sdpr0}

%LaTeX won't substitute into \index{} below, so had to use the expanded
%name.
\index{UCULIB@\emph{UCULIB}}The library described in this document, the 
\emph{\productbasenamelong{}, Version \productversion{}} (identified more 
compactly as \emph{\productbasenameshort{}-\productversion{}}, 
\emph{\productbasenameshort{}}, or \emph{\productbasenameultrashort{}}) is 
a \emph{C}/\emph{C++}-callable arithmetic/mathematical/cryptographic 
software library for microcontrollers and embedded systems.

The library includes functionality for: 

\begin{itemize}
\item \emph{Arithmetic:} native/\-large integer/\-fixed-point/\-large fixed-point/\-floating-point
      addition/\-sub\-tract\-ion/\-mul\-ti\-pli\-cat\-ion/\-division.
\item \emph{Mathematical algorithms:}
      native/\-large integer/\-fixed-point/\-large fixed-point/\-floating-point
      vector arithmetic, algebraic/transcendental functions.
\item \emph{Control system elements and linear filtering:}
      low/high-pass filtering, scaling, differentiation, integration.
\item \emph{Non-linear filtering:}
      discrete input debouncing, vertical counters.
\item \emph{Non-cryptographic hashes:}
      checksums, CRC's.
\item \emph{Cryptographic hashes:}
      hashes from the SHA-2 family (SHA-224, SHA-256, SHA-384, SHA-512, SHA-512/224, SHA-512/256).
\item \emph{Cryptography:}
      ciphers, PRNGs.
\item \emph{Bit-mapped sets:}
      union, intersection, cardinality, iteration.
\item \emph{Utility:}
      block memory
      operations, array operations.
\end{itemize}

\emph{\productbasenameshort{}} is a traditional
library in that:

\begin{itemize}
\item The source code is arranged as one function or data structure per source file.
\item The source code is compiled as one function or data structure per object file.
\item Object files are combined together into a library.
\item The linker extracts only what it needs to resolve references.  (This generally
      gives minimum FLASH and RAM footprint.)
\end{itemize}

\emph{\productbasenameshort{}} is similar in spirit to existing libraries such as
\emph{The GNU Multiple Precision Arithmetic Library}\footnote{Hereinafter called
\emph{The GMP} or \emph{GMP}.},
but it has some differences
that make it more suitable for embedded software.  The features and characteristics of the
\emph{\productbasenameshort{}} are:

\begin{itemize}
\item The library is optimized for speed (similar to \emph{The GMP}).
      (Assembly-language for common
      processor variants is provided, with a slower C fallback.  Fixed-size operands
      allow complete loop-unrolling in many cases.)
\item The library seeks to use FLASH memory as economically as possible.  (The design
      of the library allows the linker to extract the minimum required, and the code is
      compacted as much as possible without materially sacrificing speed.)
\item The library seeks to use RAM as economically as possible (but when a speed/RAM
      tradeoff exists, speed is chosen so long as the RAM cost is not unreasonable).
\item Dynamic memory allocation is not used within the library
      (unlike \emph{The GMP}).\footnote{It is still
      possible for the library to operate using dynamically-allocated memory,
      but this memory must be obtained and released by the caller.}
\item All functions in the library operate on operands of a size known by the
      caller, and this data size does not change as a result of any
      library function (unlike \emph{The GMP}).\footnote{All functions
      either accept operands of a fixed
      size characteristic of the function, or accept a measure of size
      as a formal parameter.}
\item The source code for each library variant is created from templates,
      so that the source code contains almost no C preprocessor
      switches.  (The source code is arranged in this way for
      easier debugging,\footnote{Some source-level debuggers are
      confused by complex preprocessor switches.} and to remove uncertainty about what has
      is actually being compiled or assembled.)
\item The library can be repackaged as three source files (a \emph{.h} header file, a \emph{.c}
      source file, and an assembly-language file with a target-specific file extension),
      and used as two software modules within a project rather than as a library.\footnote{The
      library source code is designed to be used in this way because some debugging tools
      are awkward at debugging libraries.}\textsuperscript{,}\footnote{These three files are created automatically
      by the generator program, but it would take typically take substantial manual editing to
      eliminate unused functions and data structures from these files so that the FLASH footprint
      is the same as if a library is used.}
\item The library is made available only as source code.\footnote{The user must
      have the software tools necessary to compile the source code and create a
      library.}
\item The library is thread-safe and core-safe.
\item The library is designed to be used on multi-core processors without
      a cache coherency protocol.\footnote{However, the library achieves this by
      requiring the caller to perform all necessary cache line invalidations
      and flushes.}
\item The library is unit tested to MC/DC coverage, and is 
      expected to contain no software defects.  
\end{itemize}


%%%%%%%%%%%%%%%%%%%%%%%%%%%%%%%%%%%%%%%%%%%%%%%%%%%%%%%%%%%%%%%%%%%%%%%%%%%%%%%
%%%%%%%%%%%%%%%%%%%%%%%%%%%%%%%%%%%%%%%%%%%%%%%%%%%%%%%%%%%%%%%%%%%%%%%%%%%%%%%
%%%%%%%%%%%%%%%%%%%%%%%%%%%%%%%%%%%%%%%%%%%%%%%%%%%%%%%%%%%%%%%%%%%%%%%%%%%%%%%
\section{Acknowledgments}
\label{ciov0:sack0}
 
TBD. 


%%%%%%%%%%%%%%%%%%%%%%%%%%%%%%%%%%%%%%%%%%%%%%%%%%%%%%%%%%%%%%%%%%%%%%%%%%%%%%%
%%%%%%%%%%%%%%%%%%%%%%%%%%%%%%%%%%%%%%%%%%%%%%%%%%%%%%%%%%%%%%%%%%%%%%%%%%%%%%%
%%%%%%%%%%%%%%%%%%%%%%%%%%%%%%%%%%%%%%%%%%%%%%%%%%%%%%%%%%%%%%%%%%%%%%%%%%%%%%%
\section{Feedback, Suggestions, and Collaboration Opportunities}
\label{ciov0:sfbk0}

I welcome all feedback and suggestions about the library and 
documentation, and I welcome all opportunities to collaborate to extend 
the functionality or supported platforms of the library.  

Please feel free to correspond with me at \texttt{dashley@\-gmail.com}.


%%%%%%%%%%%%%%%%%%%%%%%%%%%%%%%%%%%%%%%%%%%%%%%%%%%%%%%%%%%%%%%%%%%%%%%%%%%%%%%
%%%%%%%%%%%%%%%%%%%%%%%%%%%%%%%%%%%%%%%%%%%%%%%%%%%%%%%%%%%%%%%%%%%%%%%%%%%%%%%
%%%%%%%%%%%%%%%%%%%%%%%%%%%%%%%%%%%%%%%%%%%%%%%%%%%%%%%%%%%%%%%%%%%%%%%%%%%%%%%
\section{Licensing and License Interpretation}
\label{ciov0:slip0}

The library and all documentation is licensed under \emph{The MIT License}.
The license below is included in source files, other text files, and in the
LICENSE files within the project.
\emph{\productbasenameshort{}}.

\begin{small}
\begin{verbatim}
Copyright 2021 David T. Ashley (dashley@gmail.com)

Permission is hereby granted, free of charge, to any person obtaining
a copy of this software and associated documentation files
(the ``Software''), to deal in the Software without restriction,
including without limitation the rights to use, copy, modify, merge,
publish, distribute, sublicense, and/or sell copies of the Software,
and to permit persons to whom the Software is furnished to do so,
subject to the following conditions:

The above copyright notice and this permission notice shall be
included in all copies or substantial portions of the Software.

THE SOFTWARE IS PROVIDED ``AS IS'', WITHOUT WARRANTY OF ANY KIND,
EXPRESS OR IMPLIED, INCLUDING BUT NOT LIMITED TO THE WARRANTIES
OF MERCHANTABILITY, FITNESS FOR A PARTICULAR PURPOSE AND
NONINFRINGEMENT. IN NO EVENT SHALL THE AUTHORS OR COPYRIGHT HOLDERS
BE LIABLE FOR ANY CLAIM, DAMAGES OR OTHER LIABILITY, WHETHER
IN AN ACTION OF CONTRACT, TORT OR OTHERWISE, ARISING FROM, OUT
OF OR IN CONNECTION WITH THE SOFTWARE OR THE USE OR OTHER DEALINGS
IN THE SOFTWARE.
\end{verbatim}
\end{small}

My desired and intended interpretation of the license is:

\begin{itemize}
\item Source files and other text files:
      \begin{itemize}
      \item My primary goal in preserving license information within
            each source file or other text file is to make it easy
            for a reader to figure out where the content came from.
      \item The license statement contained above should be preserved
            in each 
      \end{itemize}
\end{itemize}
