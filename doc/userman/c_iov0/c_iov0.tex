\chapter{Introduction and Overview}
\label{ciov0}

%%%%%%%%%%%%%%%%%%%%%%%%%%%%%%%%%%%%%%%%%%%%%%%%%%%%%%%%%%%%%%%%%%%%%%%%%%%%%%%
%%%%%%%%%%%%%%%%%%%%%%%%%%%%%%%%%%%%%%%%%%%%%%%%%%%%%%%%%%%%%%%%%%%%%%%%%%%%%%%
%%%%%%%%%%%%%%%%%%%%%%%%%%%%%%%%%%%%%%%%%%%%%%%%%%%%%%%%%%%%%%%%%%%%%%%%%%%%%%%
\section{Overview of the Library}
\label{ciov0:sdpr0}

\label{ciov0:sdpr0}

%LaTeX won't substitute into \index{} below, so had to use the expanded
%name.
\index{UCULIB@\emph{UCULIB}}The library described in this document, the 
\emph{\productbasenamelong{}, Version \productversion{}} (identified more 
compactly as \emph{\productbasenameshort{}-\productversion{}}, 
\emph{\productbasenameshort{}}, or \emph{\productbasenameultrashort{}}) is 
a \emph{C}/\emph{C++}-callable arithmetic/mathematical/cryptographic 
software library for microcontrollers and embedded systems.

The library includes functionality for: 

\begin{itemize}
\item \emph{Arithmetic:} native/\-large integer/\-fixed-point/\-large fixed-point/\-floating-point
      addition/\-sub\-tract\-ion/\-mul\-ti\-pli\-cat\-ion/\-division.
\item \emph{Mathematical algorithms:}
      native/\-large integer/\-fixed-point/\-large fixed-point/\-floating-point
      vector arithmetic, algebraic/transcendental functions.
\item \emph{Control system elements and linear filtering:}
      low/high-pass filtering, scaling, differentiation, integration.
\item \emph{Non-linear filtering:}
      discrete input debouncing, vertical counters.
\item \emph{Non-cryptographic hashes:}
      checksums, CRC's.
\item \emph{Cryptographic hashes:}
      hashes from the SHA-2 family (SHA-224, SHA-256, SHA-384, SHA-512, SHA-512/224, SHA-512/256).
\item \emph{Cryptography:}
      ciphers, PRNGs.
\item \emph{Bit-mapped sets:}
      union, intersection, cardinality, iteration.
\item \emph{Utility:}
      block memory
      operations, array operations.
\end{itemize}

\emph{\productbasenameshort{}} is a traditional
library in that:

\begin{itemize}
\item The source code is arranged as one function or data structure per source file.
\item The source code is compiled as one function or data structure per object file.
\item Object files are combined together into a library.
\item The linker extracts only what it needs to resolve references.  (This generally
      gives minimum FLASH and RAM footprint.)
\end{itemize}

\emph{\productbasenameshort{}} is similar in spirit to existing libraries such as
\emph{The GNU Multiple Precision Arithmetic Library}, but it has these features and
characteristics so that it is more suitable for embedded software:

\begin{itemize}
\item Dynamic memory allocation is not used within the library.\footnote{It is still
      possible for the library to operate using dynamically-allocated memory,
      but this memory must be obtained and released by the caller.}
\item All functions in the library operate on operands of a size known by the
      caller, and this data size does not change as a result of any
      library function.\footnote{All functions either accept operands of a fixed
      size characteristic of the function, or accept some measure of size
      as a formal parameter.}
\item The source code for each library variant is created from templates,
      so that the source code contains almost no C preprocessor
      switches.  (The source code is arranged in this way for
      easier debugging,\footnote{Some source-level debuggers are
      confused by complex preprocessor switches.} and to remove uncertainty about what has
      is actually being compiled or assembled.)
\item The library can be repackaged as three source files (a \emph{.h} header file, a \emph{.c}
      source file, and an assembly-language file with a target-specific file extension),
      and used as two software modules within a project rather than as a library.\footnote{The
      library source code is designed to be used in this way because some debugging tools
      are awkward at debugging libraries.}
\end{itemize}

Additionally, \emph{\productbasenameshort{}} has these features and characteristics,
some shared with \emph{The GNU Multiple Precision Arithmetic Library}:

\begin{itemize}
\item The library is optimized for speed.  (Assembly-language for popular
      variants is provided, with a slower C fallback.  Fixed-size operands
      also allow complete loop-unrolling in many cases.)
\end{itemize}

\emph{\productbasenameshort{}} is made available both as binaries and as 
complete soure code.  It can be tuned, customized, and debugged at the 
source code level.  

\emph{\productbasenameshort{}} is thread-safe and core-safe.

\emph{\productbasenameshort{}} is unit tested to MC/DC coverage, and is 
expected to contain no software defects.  

